%\documentclass[12pt,cancelspace]{exam}
\documentclass[12pt,answers]{exam}
\usepackage{color,geometry}
\usepackage{verbatim}
\usepackage{amsmath, amssymb}
\usepackage{graphicx}
\geometry{hmargin={1in,1in},vmargin={1in,1in}}
\begin{document}

\section*{Multiple Choice:}
\begin{questions}
%%%%% Question 1:
\question How do you SSH into Savage?:
\begin{description}
\item[A.] Possible Answer \verb' ssh username@savage.nsm.iup.edu'
\item[B.] Possible Answer \verb' ssh savage.nsm.iup.edu@username'
\item[C.] Possible Answer \verb' username@savage.nsm.iup.edu ssh'   
\item[D.] Possible Answer \verb' savage.nsm.iup.edu@username ssh'     
%answer: C     
\end{description}
\begin{solution}
(Answer A): This is the correct syntax to connect to Savage.  
\end{solution}

%%%%% Question 2:
\question What is the name of the type of command line we have been using?:
\begin{description}
\item[A.] Possible Answer \verb' Unix'
\item[B.] Possible Answer \verb' Linux'
\item[C.] Possible Answer \verb' Bash'    
\item[D.] Possible Answer \verb' Scripting'     
%answer: C     
\end{description}
\begin{solution}
(Answer C): Bash is the type of command line we have been using .  
\end{solution}

%%%%% Question 3:
\question How do you make a file named Script.sh executable:
\begin{description}
\item[A.] Possible Answer \verb' chomod +x Script.sh'
\item[B.] Possible Answer \verb' chmod 777 Script.sh'
\item[C.] Possible Answer \verb' All of the Above'   
\item[D.] Possible Answer \verb' None of the Above'     
%answer: C     
\end{description}
\begin{solution}
(Answer C): Both A and B are correct so answer C is right.  
\end{solution}

%%%%% Question 4:
\question What is Dr. Chrispell's favorite way to make a pdf?:
\begin{description}
\item[A.] Possible Answer \verb' Word'
\item[B.] Possible Answer \verb' Google Docs'
\item[C.] Possible Answer \verb' Libre Office'   
\item[D.] Possible Answer \verb' LaTex'     
%answer: C     
\end{description}
\begin{solution}
(Answer D): We all know this answer.   
\end{solution}

\end{questions}
\section*{Short Answer}
What is the process of creating a bash script (steps taken to make one):
\begin{solution}
The first thing you must do is create the file you would like to turn into a script.

\verb touch Script.sh

You must then add a heading to the file to tell your OS what type of interperator to use for the script. we are going to use bash

\verb #!/bin/bash 

You then write the script

\verb *insert hacker script here* 

Once the script is written you want to make sure it is runnable. The easier way to do that is to do the following command 

\verb chmod +x Script.sh

or

\verb chmod 777 Script.sh 

You now can run the script using the following command 

\verb ./Script.sh 

That is how you create a bash script. 
\end{solution}
\end{document}
